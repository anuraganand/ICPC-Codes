\documentclass[landscape, a4paper, onecolumn]{article}
\usepackage{times,url,color,graphicx,epsfig}
\usepackage{amsmath, amsthm, amssymb}
\usepackage[ruled,vlined,linesnumbered,algoruled]{algorithm2e}
\renewcommand{\baselinestretch}{1.2}


\setlength{\textheight}{7.00in}
\setlength{\textwidth}{9.50in}
%\setlength{\columnsep}{0.375in}
\setlength{\topmargin}{-0.5in}
\setlength{\headheight}{0.0in}
\setlength{\headsep}{0in}
\setlength{\parindent}{0pc}
\setlength{\parskip}{0.25pc}
\setlength{\oddsidemargin}{0.00in}  % Centers text.
% \setlength{\evensidemargin}{1.00in}

\newtheorem {thm} {Theorem}
\newtheorem {dfn} {Definition}
\newtheorem {lemma} {Lemma}

\newcommand {\zero} {\sf O}
\newcommand{\bigO}[1]{\ensuremath{\mathop{}\mathopen{}\mathcal{O}\mathopen{}\left(#1\right)}}

\begin{document}
  \title{Team Notebook - Team BitBees - Indian Institute of Technology Kharagpur}
  \author{Anmol Gulati, Anurag Anand, Biswajit Paria}
  \date{}
  \maketitle
  
  \begin{verbatim}
   // Template
#include <bits/stdc++.h>
using namespace std;
typedef long long ll;
typedef pair <int,int> pii;
typedef vector <int> vi;

#define rep(i, n) for(int i = 0; i < (n); ++i)
#define forn(i, a, b) for(int i = (a); i < (b); ++i)

#define pb push_back
#define mp make_pair
#define ff first
#define ss second
#define all(c) c.begin(), c.end()
#define mset(a, v) memset(a, v, sizeof(a))
#define sz(a) ((int)a.size())
#define gi(x) scanf("%d", &x)
#define dbn cerr << "\n"
#define dbg(x) cerr << #x << " : " << (x) << " "
#define dbs(x) cerr << (x) << " "

#define foreach(c, it) for(__typeof(c.begin()) it = c.begin(); it != c.end(); ++it)

int main() {
    return 0; // Don’t forget
}

  \end{verbatim}

 \textbf{Sums}
 {\small
 \begin{eqnarray}
  \nonumber\sum_{k=0}^{n} k^4 &=& \frac{(6n^5+15n^4+10n^3-n)}{30}\\
  \nonumber\sum_{k=0}^{n} k^5 &=& \frac{(2n^6+6n^5+5n^4-n^2)}{12}\\
  \nonumber\sum_{k=0}^{n} kx^k &=& \frac{(x-(n+1)x^{n+1}+nx^{n+2}}{(x-1)^2}
 \end{eqnarray}
 }
 
 \textbf{Derangements}
 $D_n = (n-1)(D_{n-1}+D_{n-2}) = nD_{n-1}+(-1)^n$
 
 \textbf{Stirling Numbers of 1st kind}. $s_{n,k}$ is $(-1)^{n-k}$ times the number of permutations of $n$ elements with exactly $k$ permutation cycles. $|s_{n,k}| = |s_{n-1,k-1}|+(n-1)|s_{n-1,k}$. $\sum_{k=0}^n s_{n,k}x^k = x^n$. 
 
 \textbf{Stirling Numbers of 2nd kind}. $S_{n,k}$ is the number of ways to partition a set of $n$ elements  with exactly $k$ non-empty subsets. $S_{n,k} = S_{n-1,k-1}+kS_{n-1,k}$. $S_{n,1} = S_{n,n} = 1$. $\sum_{k=0}^n S_{n,k}x^k = x^n$.
 
 \textbf{Chinese Remainder Theorem}
 System $x \equiv a_i \pmod{m_i}$ for $i = 1,\cdots,n$ with pairwise relatively prime $m_i$ has a unique solution modulo $M = m_1m_2\cdots m_n: x = \sum_{i=1}^{n} a_ib_i\frac{M}{m_i} \pmod{M}$
 
 System $x \equiv a \pmod{m}, x \equiv b \pmod{m}$, has solution iff $a \equiv b \pmod{g}$, where $g = \gcd(m,n)$. The solution is unique modulo $L = \frac{mn}{g}$, and equals $x \equiv a + T(b-a)m/g \equiv b + S(a-b)n/g \pmod{L}$, where $S$ and $T$ are integer solutions of $mT + nS = \gcd(m,n)$.
 
 \textbf{Mobius Function}
 $\mu(1) = 1$. $\mu(n) = 0$, if $n$ is not square free. $\mu(n) = (-1)^s$, if $n$ is the product of $s$ distinct primes. Let $f, F$ be functions on positive integers. If $ \forall n \in \mathbb{N}, F(n) = \sum_{d|n} f(d)$, then $f(n) = \sum_{d|n} \mu(d)F(n/d)$ and vice versa. $\phi(n) = \sum_{d|n} \mu(d)(n/d)$. $\sum_{d|n} \mu(d) = 1$. If $f$ is multiplicative, then $\sum_{d|n} \mu(d)f(d) = \prod_{p|n}(1-f(p), \sum_{d|n} \mu(d)^2 f(d) = \prod_{p|n} (1+f(p)$.
 
 \textbf{Postage stamp problem}
 Let $a, b$ be relatively prime integers. There are exactly $\frac{1}{2}(a-1)(b-1)$ numbers not of the form $ax + by$ where $x,y \ge 0$ and the largest is $(a-1)(b-1)-1$.
 
 \textbf{Fermat's two square theorem}
 Odd prime $p$ can be represented as sum of two squares iff $p \equiv 1 \pmod{4}$. Product of two sums of two squares is a sum of two squares. Thus $n$ is a sum of two square iff every prime of the form $ p = 4k+3$ occurs an even number of times in the factorization of $n$.
 
 \textbf{Pick's Theorem}
 $I = A - B/2 +1$ where $A$ is the area of a lattice polygon, $I$ is number of lattice
points inside it, and $B$ is number of lattice points on the boundary. Number of lattice points minus
one on a line segment from $(0,0)$ and $(x,y)$ is $\gcd(x,y)$.

  \textbf{Angular Bisector} of $\angle ABC$ is line $BD$, where $D = \frac{BA}{|BA|} + \frac{BC}{|BC|}$. Center of incircle of $\Delta ABC$ is $\frac{a}{a+b+c} A + \frac{b}{a+b+c} B + \frac{c}{a+b+c} C$. Radius of incircle = $\frac{2\Delta}{a+b+c}$.
  
\end{document}
